\documentclass{exercice}

\renewcommand{\coursename}{foo}
\renewcommand{\coursecode}{bar}
\renewcommand{\degreename}{foobar}
%% http://www.ntg.nl/EuroTeX/2012/talks/2arnold.pdf
%\gdef\conditionmacro{1}

\ifx\conditionmacro\undefined
  \immediate\write18{%
    pdflatex --jobname="\jobname-examA"
    "\gdef\string\conditionmacro{1}\string\PassOptionsToClass{exam,noanswer}{exercice}\string\input\space\jobname\string"
  }%
  \immediate\write18{%
    pdflatex --jobname="\jobname-examB"
    "\gdef\string\conditionmacro{2}\string\PassOptionsToClass{exam,noanswer,versionB}{exercice}\string\input\space\jobname\string"
  }%
  \immediate\write18{%
    pdflatex --jobname="\jobname-corrA"
    "\gdef\string\conditionmacro{3}\string\input\space\jobname"
  }%
  \immediate\write18{%
    pdflatex --jobname="\jobname-corrB"
    "\gdef\string\conditionmacro{4}\string\PassOptionsToClass{versionB}{exercice}\string\input\space\jobname\string"
  }%
  \expandafter\stop
 \fi


\title{Test}
\author{}
\begin{document}

\maketitle

\begin{Exercise}[title={Système positionnel}]
  Considérons une base quelconque $b$ de chiffres $c_i$.

  \versions{version A}{version B}
\Question 
Comment s'écrivent les nombres 0 et 1 ?
\ExamLines{1}
\Question
Quel est l’intervalle pour $i$ ?
\ExamLines{1}
\Question
Comment s'écrit le nombre $b$ ? Et ses puissances $b^n$ ?
\ExamLines{1}
\Question
Quel est le plus petit nombre à 4 chiffres ? Et le plus grand ?
\ExamLines{1}
\ExamGrid{10}
\end{Exercise}

\begin{Answer}
  \Question 
  0 et 1 ;)  
  \\
  La réponse est $c_0$ et $c_1$.
  \Question
  $[0, b[$
  \Question
  $b=10$ ou en d'autres termes $b=c_1c_0$.
  \\
  $b^n$ $c_1$ suivi de $n$ digits $c_0$. 
  \Question
  $c_1c_0c_0c_0$ (plus petit) et   $c_{b-1}c_{b-1}c_{b-1}c_{b-1}$ (plus grand).
\end{Answer}

\ExamNewpage
\ExamOnly{\textbf{Exam Only}}

\ifnum\conditionmacro=1 Condition is 1\fi 

\ifnum\conditionmacro=2 Condition is 2\fi

\ifnum\conditionmacro=3 Condition is 3\fi

\verb|\conditionmacro| is \texttt{\meaning\conditionmacro}.



\clearpage 
\begin{Exercise}[title={Système positionnel II}]
  Considérons une base quelconque $b$ de chiffres $c_i$.

  \versions{version A}{version B}
\Question 
Comment s'écrivent les nombres 0 et 1 ?
\ExamLines{2}
\Question
Quel est l’intervalle pour $i$ ?
\ExamLines{2}
\Question
Comment s'écrit le nombre $b$ ? Et ses puissances $b^n$ ?
\ExamLines{1}
\Question
Quel est le plus petit nombre à 4 chiffres ? Et le plus grand ?
\ExamLines{2}
% \ExamGrid{10}
\end{Exercise}

\begin{Answer}
  \Question 
  0 et 1 ;)  
  \\
  La réponse est $c_0$ et $c_1$.
  \Question
  $[0, b[$
  \Question
  $b=10$ ou en d'autres termes $b=c_1c_0$.
  \\
  $b^n$ $c_1$ suivi de $n$ digits $c_0$. 
  \Question
  $c_1c_0c_0c_0$ (plus petit) et   $c_{b-1}c_{b-1}c_{b-1}c_{b-1}$ (plus grand).
\end{Answer}

\clearpage

\begin{Exercise}[title={Système positionnel II}]
  Considérons une base quelconque $b$ de chiffres $c_i$.

  \versions{version A}{version B}
\Question 
Comment s'écrivent les nombres 0 et 1 ?
\ExamLines{2}
\Question
Quel est l’intervalle pour $i$ ?
\ExamLines{2}
\Question
Comment s'écrit le nombre $b$ ? Et ses puissances $b^n$ ?
\ExamLines{1}
\Question
Quel est le plus petit nombre à 4 chiffres ? Et le plus grand ?
\ExamLines{2}
% \ExamGrid{10}
\end{Exercise}

\begin{Answer}
  \Question 
  0 et 1 ;)  
  \\
  La réponse est $c_0$ et $c_1$.
  \Question
  $[0, b[$
  \Question
  $b=10$ ou en d'autres termes $b=c_1c_0$.
  \\
  $b^n$ $c_1$ suivi de $n$ digits $c_0$. 
  \Question
  $c_1c_0c_0c_0$ (plus petit) et   $c_{b-1}c_{b-1}c_{b-1}c_{b-1}$ (plus grand).
\end{Answer}
\clearpage
\end{document}
%%% Local Variables:
%%% mode: latex
%%% TeX-master: t
%%% End:
