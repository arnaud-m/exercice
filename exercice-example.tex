%\documentclass[french,noanswer,exam,nograding]{exercice}
\documentclass[french]{exercice}
\ExamExport
\usepackage[T1]{fontenc}

\renewcommand{\coursename}{My Course}
\renewcommand{\degreename}{Degree}
\renewcommand{\ExamDuration}{2 heures}
\renewcommand{\ExamRemark}{Remarque \dots }

\title{Test}
\author{}


\begin{document}

\maketitle
\listofexercises


\begin{Exercise}[title={Système positionnel}]

  Considérons une base quelconque $b$ de chiffres $c_i$.

  \versions{version A}{version B}
\Question 
Comment s'écrivent les nombres 0 et 1 ?
\ExamLines{1}
\Question
Quel est l’intervalle pour $i$ ?
\ExamLines{1}
\Question
Comment s'écrit le nombre $b$ ? Et ses puissances $b^n$ ?
\ExamLines{1}
\Question
Quel est le plus petit nombre à 4 chiffres ? Et le plus grand ?
\ExamLines{1}
\ExamGrid{10}
\end{Exercise}

\begin{Answer}
  \Question 
  0 et 1 ;)  
  \\
  La réponse est $c_0$ et $c_1$.
  \Question
  $[0, b[$
  \Question
  $b=10$ ou en d'autres termes $b=c_1c_0$.
  \\
  $b^n$ $c_1$ suivi de $n$ digits $c_0$. 
  \Question
  $c_1c_0c_0c_0$ (plus petit) et   $c_{b-1}c_{b-1}c_{b-1}c_{b-1}$ (plus grand).
\end{Answer}

\ExamNewpage
\ExamOnly{\textbf{Exam Only}}

% \ifnum\ExamExportKey=1 Condition is 1\fi 

% \ifnum\ExamExportKey=2 Condition is 2\fi

% \ifnum\ExamExportKey=3 Condition is 3\fi

\verb|\ExamExportKey| is \texttt{\meaning\ExamExportKey}.



\clearpage 
\begin{Exercise}[title={Système positionnel II}]
  Considérons une base quelconque $b$ de chiffres $c_i$.

  \versions{version A}{version B}
\Question 
Comment s'écrivent les nombres 0 et 1 ?
\ExamLines{2}
\Question
Quel est l’intervalle pour $i$ ?
\ExamLines{2}
\Question
Comment s'écrit le nombre $b$ ? Et ses puissances $b^n$ ?
\ExamLines{1}
\Question
Quel est le plus petit nombre à 4 chiffres ? Et le plus grand ?
\ExamLines{2}
% \ExamGrid{10}
\end{Exercise}

\begin{Answer}
  \Question 
  0 et 1 ;)  
  \\
  La réponse est $c_0$ et $c_1$.
  \Question
  $[0, b[$
  \Question
  $b=10$ ou en d'autres termes $b=c_1c_0$.
  \\
  $b^n$ $c_1$ suivi de $n$ digits $c_0$. 
  \Question
  $c_1c_0c_0c_0$ (plus petit) et   $c_{b-1}c_{b-1}c_{b-1}c_{b-1}$ (plus grand).
\end{Answer}

\clearpage

\begin{Exercise}[title={Système positionnel II}]
  Considérons une base quelconque $b$ de chiffres $c_i$.

  \versions{version A}{version B}
\Question 
Comment s'écrivent les nombres 0 et 1 ?
\ExamLines{2}
\Question
Quel est l’intervalle pour $i$ ?
\ExamLines{2}
\Question
Comment s'écrit le nombre $b$ ? Et ses puissances $b^n$ ?
\ExamLines{1}
\Question
Quel est le plus petit nombre à 4 chiffres ? Et le plus grand ?
\ExamLines{2}
% \ExamGrid{10}
\end{Exercise}

\begin{Answer}
  \Question 
  0 et 1 ;)  
  \\
  La réponse est $c_0$ et $c_1$.
  \Question
  $[0, b[$
  \Question
  $b=10$ ou en d'autres termes $b=c_1c_0$.
  \\
  $b^n$ $c_1$ suivi de $n$ digits $c_0$. 
  \Question
  $c_1c_0c_0c_0$ (plus petit) et   $c_{b-1}c_{b-1}c_{b-1}c_{b-1}$ (plus grand).
\end{Answer}
\clearpage


  \begin{Exercise}[title={Système positionnel II}]
  \end{Exercise}
  
  \begin{Exercise}[title={Système positionnel II}]
  \end{Exercise}
  
  \begin{Exercise}[title={Système positionnel II}]
  \end{Exercise}
  
  \begin{Exercise}[title={Système positionnel II}]
  \end{Exercise}
  
\end{document}
%%% Local Variables:
%%% mode: latex
%%% TeX-master: t
%%% End:
